\input{preambuloSimple.tex}

%------------------------------------------------------------------------
%	TÍTULO Y DATOS DEL ALUMNO
%------------------------------------------------------------------------

\title{	
\normalfont \normalsize 
\textsc{\textbf{Modelos de Computación (2016-2017)} \\ Grado en Ingeniería Informática \\ Universidad de Granada} \\ [25pt] % Your university, school and/or department name(s)
\horrule{0.5pt} \\[0.4cm] % Thin top horizontal rule
\huge Autómatas Celulares \\ % The assignment title
\horrule{2pt} \\[0.5cm] % Thick bottom horizontal rule
\author{Elena María Gómez Ríos y Jose Luis Martínez Ortiz}
}

\date{\normalsize\today} % Incluye la fecha actual

%-------------------------------------------------------------------------
% DOCUMENTO
%-------------------------------------------------------------------------

\begin{document}

\maketitle % Muestra el Título

\newpage

%-------------------------------------------------------------------------
%	Resumen introductorio (entre 5 y 15 lineas)
%----------------------------------------------------------------------
\begin{abstract} 
Se ha realizado un trabajo sobre los Autómatas Celulares desarrollados en la época de los 60 por \textit{Von Neumann}. La teoría de los Autómatas Celulares ha ido cambiando desde sus inicios pasando por \textit{Wolfram} quien definió una reglas para un tipo en concreto y definió cuatro clases de Autómatas Celulares. Se describen algunas de las reglas más utilizadas y algunas de las aplicaciones actuales para estos modelos como el modelamiento del flujo de tráfico en carreteras. Y por ultimo se le da un enfoque actual con la utilización de algoritmos genéticos para desarrollar Autómatas Celulares.

% IDEA BASICA Y GENERAL - HAY QUE HACERLO BIEN!!!!!!!!

\end{abstract}

%--------------------------------------------------------------------------
%	Introducción
%-------------------------------------------------------------------------

\section{Introducción} % 
% http://web.archive.org/web/20080907225701/http://yupana.autonoma.edu.co/publicaciones/yupana/005/autocelular/Automatas.html

%

Para poder entender que son los Autómatas Celulares primero debemos comprender de donde viene la necesidad de crearlos. En la historia del ser humano y las ciencias se ha perseguido el modelamiento del mundo y todos sus fenómenos que ocurren en el, en concreto de los sistemas físicos, eléctricos y mecánicos mediante el uso de los modelos matemáticos. Matemáticamente hablando, estos fenómenos de la naturaleza son sistemas de naturaleza continua y han sido tratados con ecuaciones diferenciales, integrales funcionales y variables de estado entre otros procedimientos matemáticos para su modelamiento. También están los modelos aproximados y discretos que ofrecen una teoría muy cercana a la realidad con la ventaja de utilizar valores finitos. Para ello se ha utilizado la discretización y la digitalización de sistemas.\\

Una de las técnicas matemáticas complejas para el modelado de sistemas físicos y mecánicos es el \textit{Método de los Elementos Finitos} (FEM), cuya finalidad es discretizar espacios de naturaleza continua, sobre los cuales es posible realizar análisis numéricos para comprender, por medio de un modelo discreto, el comportamiento de sistemas analógicos. Pero esta técnica resulta muy compleja de aplicar por su dificultad para poder lograr modelos que describan sus comportamientos de forma precisa. Esta técnica (FEM) tiene una alta aplicación en el análisis de sistemas y espacios físicos-mecánicos donde el objetivo del modelo es comprender el comportamiento del sistema en el ámbito de la resistencia de los materiales, la dinámica de partículas y en general el comportamiento con la interacción  de los elementos base del sistema en el espacio donde reside.\\

Aun así hay todavía un amplio grupo de sistemas que es imposible modelar con estas técnicas debido a diversos motivos, como por ejemplo, sistemas químicos, biológicos, evolutivos, eléctricos, computacionales e inclusive otros sistemas físicos y mecánicos. Para estos sistemas que no se podían modelar con FEM han aparecido a lo largo de la historia diferentes técnicas para obtener su modelo continuo, una de ellas fue el modelado con \textit{Autómatas Celulares}.




\section{Autómatas Celulares}
Aunque no existe una definición formal sobre que es un Autómata Celular entendemos que es un estudio de modelado discreto para un sistema que evoluciona en generaciones o iteraciones discretas. Es recomendable utilizar esta técnica cuando se tiene un sistema con una colección masiva de objetos simples que interactúan unos con otros de forma aleatoria y por ello es utilizado en la teoría de la computación, las matemáticas, la física, las ciencias complejas, etc. \\

Un Autómata Celular consiste en una rejilla regular de celdas, donde cada celda se conoce como célula y representa un estado del conjunto de estados disponibles del sistema. Cada celda puede tomar un valor de un rango de valores definidos para este sistema en particular, siempre que el valor sea discreto y perteneciente al conjunto de los enteros. Además cada celda viene definida por su ``vecindario'', entendemos por vecindario al conjunto finito de las células adyacentes a una célula en concreto.  La rejilla va cambiando en el tiempo y actualizando el estado de sus células, a cada instante de tiempo se le denomina generación y el estado de las células en una generación no varía.  \\


En el autómata celular cuando se avanza de generación, se actualiza el valor de todas las células del autómata aplicando una función de transición (``evolución''), que avanza el autómata al estado siguiente. Esta función de transición viene determinada por una ecuación matemática que toma como argumentos los valores de la vecindad de la célula además de del valor de la propia célula. Siempre se aplica de forma homogénea y para cada paso discreto del tiempo.\\

Una manera de simular un autómata celular bidimensional es con una cuadrícula de tamaño finito. Cada célula tiene dos posibles estados, viva o muerta. La vecindad de la célula se define por una regla en concreto, ya que la forma de considerar la adyacencia varía dependiendo del modelo y la versión del autómata que se utilice. Los tipos más comunes de vecindad son los que definieron \textit{Neumann} y \textit{Moore} y que se apodaron igual que sus autores. El vecindario definido por \textit{Von Neumann} se define como el conjunto de cuatro células que rodean ortogonalmente a una célula central. Hay una variante de este vecindario llamada ``zona ampliada de \textit{Von Neumann}'', que consiste en ampliar el vecindario a las ocho células que rodean ortogonalmente a la célula central. El vecindario definido por \textit{Moore} consiste en las ocho células adyacentes que rodean a una célula central.\\


\begin{figure}[H]
\centering
\includegraphics[scale=0.5]{imagenes/CA_Neumann.png}
\hspace{2cm}
\includegraphics[scale=0.5]{imagenes/CA-Moore.png}
\caption{Vecindad en color rojo de la célula azul de forma gráfica, izquierda vecindad ampliada de Von Neumann y derecha vecindad de Moore.}
\label{fig:vecindad}

\end{figure}

La ecuación general del sistema de reglas que se utiliza para determinar el estado de la célula en la generación siguiente es del tipo $k^{k^s}$ donde $k$ es el número de posibles estados para una célula, y $s$ es el número de células de la vecindad (incluyendo al propia célula). De esta forma con la vecindad de \textit{Moore} en un sistema de dos dimensiones sería de $2^{2^9}$ el número total de autómatas posibles.\\

Los Autómatas Celulares se representaban con rejillas finitas en vez de las rejillas infinitas, ya que con la rejilla infinita no se podría realmente trabajar de forma precisa. Pero al utilizar una rejilla finita se presentaba una serie de problemas, el primero de ellos es el tratamiento que se realiza con las células de los bordes. Las células situadas en los limites de la rejilla ya no cuentan con una vecindad igual que el resto de las células del autómata, de esta forma la función de transición ya no trataría de la misma manera a todas las células. Si aplicamos una función  distinta a las células de los bordes no se estaría tratando tampoco de forma igualitaria a todas las células del autómata. Para este problema se han establecido unas soluciones llamadas ``condiciones frontera'', cada una de ellas esta orientada a solucionar distintos problemas reales del modelado:
\begin{itemize}
\item Frontera Abierta: Se considera que fuera de la rejilla residen células, todas con un valor fijo. En el caso particular de que el autómata tenga dos estados en su conjunto $k$, una frontera se dice fría si las células fuera de la frontera se consideran muertas, y caliente si se consideran vivas.
\item Frontera Periódica: Se considera a la rejilla como si sus extremos se tocaran. En una rejilla de dimensión 1, esto puede visualizarse en dos dimensiones como una circunferencia. En tres dimensiones la rejilla podría visualizarse como un toroide.

\begin{figure}[H]
\centering
\includegraphics[scale=0.25]{imagenes/Torus.png}
\caption{Toroide, rejilla de Frontera Periodica en tres dimensiones.}
\label{fig:torus}
\end{figure}

\item Frontera Reflectora:  Se considera que las células fuera de la rejilla ``reflejan'' los valores de aquellas dentro de la rejilla. Así, una célula que estuviera junto al borde de la rejilla, por la parte de fuera tomaría como valor el de la célula que esté junto al borde de la rejilla, dentro de ella. 

 \item Sin Frontera: Haciendo uso de implementaciones que hagan crecer dinámicamente el uso de memoria de la rejilla implementada, se puede asumir que cada vez que las células deben interactuar con células fuera de la rejilla, esta se hace más grande para poder dar cabida a estas interacciones. Obviamente, existe un límite (impuesto por la memoria disponible) para esta condición. Es muy importante no confundir esta condición de frontera con la definición original de autómata celular cuya rejilla es inicialmente infinita. En el caso de un autómata celular sin frontera, la rejilla comienza con un tamaño definido y finito, y conforme se requiera va creciendo en el tiempo, lo cual no lo hace necesariamente un modelo más cercano a la realidad, pues si se inicializara la rejilla aleatoriamente, con esta condición sólo se pueden inicializar las células dentro de la rejilla inicial finita, mientras que en el caso de la definición original, en teoría todas las células de la rejilla infinita deberían ser inicializadas.
\end{itemize}


\subsection{Autómatas Celulares Lineales}
Unos de los Autómatas Celulares más utilizados y comunes son los denominados Autómatas Celulares lineales. Llamados así por que utilizan solo una dimensión, es decir, sus células están dispuestas una a continuación de otra a modo de una cadena. Si el autómata celular lineal consta de n células, cada una de ellas se nombrará por $\langle i \rangle$ con $ 0  \leq i \leq n-1 $. Por ejemplo, un autómata celular lineal con cinco células sería:

\begin{figure}[H]
\centering
\includegraphics[scale=0.7]{imagenes/name_cell.png}
\label{fig:lineal_1}
\end{figure}

Si, además, $S_k$ es el conjunto de $k$ estados y $a_i^t \in S_k, \enspace 0 \leq i \leq n-1$, es el estado de la célula $\langle i \rangle$ en e instante $t$, entonces se denomina ``\textit{configuración del Autómata Celular en el instate $t$}'' y se denota por $C^t$ al siguiente vector:
$$C^t = \enspace \big( a_0^t,a_1^t, ..., a_{n-1}^t \big) \in S_k , ... n ... , S_k$$

La evolución de un autómata celular a lo largo del tiempo se representa de forma sencilla sin más que escribir las sucesivas configuraciones de sus células, una debajo de otra, a lo que llamaremos diagrama de evolución del autómata celular. A continuación se muestra un ejemplo del diagrama de evolución de un autómata celular de 5 células.

\begin{table}[H]
\centering
\begin{tabular}{lllllll}
\cline{1-5}
\multicolumn{1}{|l|}{$a_0^0$} & \multicolumn{1}{l|}{$a_0^1$} & \multicolumn{1}{l|}{$a_0^2$} & \multicolumn{1}{l|}{$a_0^3$} & \multicolumn{1}{l|}{$a_0^4$} & $\rightarrow$ & $C^0$ \\ \cline{1-5}
\multicolumn{1}{|l|}{$a_1^0$} & \multicolumn{1}{l|}{$a_1^1$} & \multicolumn{1}{l|}{$a_1^2$} & \multicolumn{1}{l|}{$a_1^3$} & \multicolumn{1}{l|}{$a_1^4$} & $\rightarrow$ & $C^1$ \\ \cline{1-5}
\multicolumn{1}{|l|}{$a_2^0$} & \multicolumn{1}{l|}{$a_2^1$} & \multicolumn{1}{l|}{$a_2^2$} & \multicolumn{1}{l|}{$a_2^3$} & \multicolumn{1}{l|}{$a_2^4$} & $\rightarrow$ & $C^2$ \\ \cline{1-5}
 &  & \multicolumn{1}{c}{...} &  &  &  & \multicolumn{1}{c}{...}
\end{tabular}
\end{table}

Denotaremos por $V_i$ a la vecindad de la célula i-ésima, es decir al conjunto de células cuyo estado va a influir en el de la célula $i$ según la regla de transición que se considere. La vecindades más comunes en los autómatas celulares lineales son de carácter simétrico, de modo que la célula i-ésima es la célula central. Estas vecindades pueden escribirse de la siguiente manera:
$$V_i (r) = \lbrace (i-r), ... ,(i-1),(i),(i+1), ... ,(i+r)\rbrace$$

donde $r$ recibe el nombre de ``radio de la vecindad''. Existen otros tipos de vecindades no simétricas como las arbitrarias que tiene la forma:
$$V_i = \lbrace (i-u), (i), (i+v) \rbrace \mid u,v \in \mathbb{N} \enspace y \enspace u \ne v$$

Dada una célula $i$, con $i < r$ ó $i > n-r $, la determinación de la vecindad $V_i(r)$ queda restringida a determinadas condiciones de contorno del autómata celular lineal.

\section{Historia}
La historia de los Autómatas Celulares se remonta a la década de los 40, cuando fue desarrollada su teoría por \textit{Von Neumann} y descrita en su libro ``The Theory of Self-reproducing Automata'' \cite{Teoria_Von_neumann}.
A partir de entonces la teoría original ha vivido tres grandes etapas por grandes matemáticos/físicos/informáticos que aportaron cada uno de ellos un nuevo enfoque o una nueva característica importante en la teoría de los Autómatas Celulares.
Vamos a describir como fue la evolución en estas tres etapas de forma cronológica. 

%--------------------------------------------------------------------------
%	Era de Von Neumann
%--------------------------------------------------------------------------

\subsection{Era de Von Neumann} % 
John Von Neumann trabajaba en el Laboratorio Nacional Los Álamos junto a su amigo físico y científico de la computación Stanislaw Ulam. Neumann quería desarrollar robots  
\begin{wrapfigure}{l}{0.4\linewidth}
\centering
\includegraphics[scale=3]{imagenes/neumann.png}
\caption{John Von Neumann}
\label{fig:neumann}
\end{wrapfigure}
que pudieran autoreplicarse, es decir, robots que se pudiera reproducirse a si mismos, pero pronto se dio cuenta de la altísima dificultad que suponía crear un robot que pudiera crear a su vez otros robots. La máquina tendría un coste de prestaciones muy elevado y requeriría una infinidad de piezas para poder construirse. Ulam que estaba estudiando el crecimiento de los cristales utilizando una red de celosía como modelo, ayudo a Neumann con el problema que se le había presentado sugiriéndole que utilizara un sistema discreto para crear un modelo reduccionista de autoreplicación. Además Neumann leyó un documento de Simposio Hixon titulado ``The general and logical theory of automata'' el cual le sirvió para desarrollar y publicar su teoría de Autómatas Celulares en su trabajo \cite{Teoria_Von_neumann} a finales de los años 40. \\

Esta primera teoría les sirvió a Ulam y Neumann para crear un método que calcula el movimiento de liquido. El concepto fundamental del método fue tratar el liquido como un conjunto de unidades discretas, finito y muy grande de pequeños elementos más simples cuyo comportamiento dependía de los elementos que lo rodeaban de esta forma se creó el primer autómata celular en la década de los cincuenta. Al igual que la red de celosía que estudiaba Ulam, los Autómatas Celulares de Neumann son bidimensionales, con su autoreplicación implementado de forma algorítmica. El resultado fue un autómata celular que en su interior creaba copias dentro de su sistema y con 29 estados por célula, de esta forma dio una prueba de la existencia de un patrón que podría hacer infinidad de copias de sí mismo dentro del su propio universo de células. Este diseño se le llamo ``constructor universal de Von Neumann''.\\

A partir de entonces los Autómatas Celulares fueron amplia mente estudiados por por varios científicos de la época.
En la década de 1960, fueron estudiados como un tipo particular de sistema dinámico y se estableció también una conexión con el campo matemático de la dinámica simbólica. En 1969, Gustav A. Hedlund realizo una compilación de estudios matemáticos de los Autómatas Celulares, un documento fundamental a partir de entonces. El resultado más importante fue la caracterización en el teorema de Curtis-Hedlund-Lyndon del conjunto de reglas globales de los Autómatas Celulares como el conjunto de continuos endomorfismos de espacios de turno.\\

También en 1969 el informático Alvy Ray Smith\cite{alvy} completó una tesis doctoral de Stanford en Teoría de Autómatas Celulares, el primer tratamiento matemático de los Autómatas Celulares enfocado al mundo de la ciencia de la computación y a los ordenadores en general. Muchos artículos posteriores vinieron de esta tesis: Mostró la equivalencia de los vecindarios de diversas formas, como reducir una vecindad de Moore a una vecindad de Von Neumann o cómo reducir cualquier zona de una vecindad de Von Neumann.


%--------------------------------------------------------------------------
%	Era de John Horton Conway
%--------------------------------------------------------------------------

\subsection{Era de John Horton Conway} % 
Los autómatas celulares tuvieron una gran expansión gracias a \textit{John Horton Conway} que en 1970 dio a conocer el autómata celular que probablemente sea el más conocido: el Juego de la vida (Life), publicado por Martin Gardner en su columna ``Mathematical Games'' en la revista ``Scientific American''. El Juego de la Vida consiste en una cuadrícula bidimensional con dos estados donde se coloca al inicio un patrón de células ``vivas'', representadas por una celda de color negro o ``muertas'', representadas por una celda de color blanco. 

\begin{figure}[H]
\centering
\includegraphics[scale=0.6]{imagenes/life_1.png}
\hspace{2cm}
\includegraphics[scale=0.6]{imagenes/life_2.png}
\caption{Ejemplos de patrones del Juego de la Vida}
\label{fig:patrones}
\end{figure}


La vecindad para cada célula son ocho: los vecinos formados por la vecindad de  Moore. De manera repetida, se aplican simultáneamente sobre todas las células de la cuadrícula las siguientes 3 reglas:

\textit{Nacimiento}: se reemplaza una célula muerta por una viva si dicha célula tiene exactamente 3 vecinos vivos.\\
\textit{Muerte}: se reemplaza una célula viva por una muerta si dicha célula no tiene más de 1 vecino vivo (muerte por aislamiento) o si tiene más de 3 vecinos vivos (muerte por sobrepoblación).\\
\textit{Supervivencia}: una célula viva permanecerá en ese estado si tiene 2 o 3 vecinos vivos.\\

El juego hizo inmediatamente famoso Conway, pero también abrió un nuevo campo de investigación con los autómatas celulares. Debido a las analogías de la vida con el ascenso, caída y transformaciones de una sociedad de organismos vivos, que pertenece a una creciente clase de juegos denominados "juegos de simulación" (juegos que se asemejan a los procesos de la vida real).\\

A pesar de su simplicidad, el sistema logra una impresionante diversidad de comportamientos, fluctuando entre aparente aleatoriedad y el orden. Una de las características más evidentes de juego de la vida es la frecuente aparición de planeadores o ``Gliders'', que son un tipo de patrón que se desplaza por la cuadrícula de forma continua , los arreglos de células que se mueven esencialmente a sí mismos a través de la rejilla. Es posible disponer el autómata para que los planeadores interactúen para realizar cálculos, y después de mucho esfuerzo se ha demostrado que el juego de la vida puede emular  la máquina de Turing.

\begin{figure}[H]
\centering
\includegraphics[scale=1]{imagenes/planeador.png}
\caption{Patrón planeador en sus cuatro generaciones}
\label{fig:planeador}
\end{figure}



%--------------------------------------------------------------------------
%	Era de Stephen Wolfram
%--------------------------------------------------------------------------
\newpage
\subsection{Era de Stephen Wolfram} % 
\textit{Stephen Wolfram} comenzó a trabajar de forma independiente en autómatas celulares a mediados de 1981, después de considerar cómo los patrones complejos aparecían en la naturaleza. Sus investigaciones fueron impulsadas inicialmente por un interés en los  
\begin{wrapfigure}{l}{0.4\linewidth}
\centering
\includegraphics[scale=0.3]{imagenes/stephen.png}
\caption{Stephen Wolfram}
\label{fig:Stephen}
\end{wrapfigure}
sistemas de modelado, tales como las redes neuronales. Él publicó su primer artículo en ``Reviews of Modern Physics'' donde investiga los autómatas celulares elementales ( Regla 30 en particular) en junio de 1983. La inesperada complejidad del comportamiento de estas reglas simples llevó a \textit{Wolfram} a sospechar que la complejidad en la naturaleza puede ser debido a mecanismos similares. Durante este período de \textit{Wolfram} formuló los conceptos de la intrínseca aleatoriedad y la irreductibilidad computacional , y sugirió que la regla 110 puede ser universal, hecho fue demostrado por asistente de investigación Mateo Cook, en la década de 1990. Durante su investigación tras múltiples simulaciones de diferentes Autómatas Celulares, estableció la siguiente clasificación de los mismos en virtud del comportamiento manifestado en los diagramas de evolución \cite{wolfram}:
\begin{itemize}
\item \textit{Autómata Celular de Clase 1}: son aquellos que evolucionan a estados homogéneos o constantes, es decir, o todo ceros, o todo unos. Además, dicha evolución es independiente de la configuración inicial considerada.
\item \textit{Autómata Celular de Clase 2}: son los autómatas que dan lugar a conjuntos de estructuras periódicas y estables. En ellos la evolución del estado de una determinada célula a lo largo del tiempo estará influida por los estados de un grupo fijo de células de la configuración inicial.
\item \textit{Autómata Celular de Clase 3}: son todos aquellos autómatas celulares cuyo comportamiento se vuelve caótico con el paso del tiempo, de tal forma que el cambio de estado de una célula va a depender cada vez más de una mayor número de estados iniciales. Se ha conjeturado que el cálculo de dichos estados se puede hacer mediante un simple algoritmo
\item \textit{Autómata Celular de Clase 4}: son aquellos que dan lugar a estructuras complejas, las cuales pueden permanecer localizadas en el espacio o bien moverse a lo largo del mismo. Es esta clase, la evolución del estado de una célula en particular dependerá de una gran cantidad de estados iniciales, de manera que la determinación exacta de dicho estado es un problema cuya complejidad es equivalente a la propia simulación explicita del Autómata Celular.
\end{itemize}

\begin{figure}[H]
\centering
\includegraphics[scale=0.7]{imagenes/AutomatasCelulares_WolframClasses.png}
\caption{Ejemplos de Autómatas Celulares de cada clase.}
\label{fig:clase}
\end{figure}

En la figura superior se puede apreciar claramente de forma visual los tipos de clases que estableció \textit{Wolfram}


%--------------------------------------------------------------------------
%	Reglas de AC 
%--------------------------------------------------------------------------

\section{Reglas del Autómata} % 
Wolfram también definió un sistema de reglas o de notación consistente en asignar un numero natural de 0 a 255 a un Autómata Celular de la siguiente forma.\\

Considerando el caso particular de los Autómatas Celulares definidos de modo que su conjunto de estados es $S = \mathbb{Z}_2$ y el radio de la vecindad es $r = 1$. Para este caso particular de un Autómata Celular, el número de reglas de transición existentes es, según se menciono en la definición, de $2^{2^{2r+1}} = 2^8 = 256$ reglas de transición. Dado que $r = 1$ la vecindad de una célula está formada por ella misma, la célula a su izquierda y la célula situada a su derecha. Como el conjunto de estados es $\mathbb{Z}_2$, los dos estados en que puede encontrarse una célula son 0 ó 1. Consecuentemente existen $2^3 = 8$ posibles configuraciones de la vecindad de una célula dada, a saber: \\
$$111 \qquad 110 \qquad 101 \qquad 100 \qquad 011 \qquad 010 \qquad 001 \qquad 000$$

El primer dígito de cada configuración representa el estado de la célula de la izquierda, el dígito centra el estado de la célula de cuya vecindad hablamos y el último dígito hace referencia al estado de la célula de la derecha. Es claro que toda regla de transición consiste en asignar a cada una de estas posibles configuraciones de la vecindad un elemento de $\mathbb{Z}_2$. El número a asignar a una regla de transición se calculará de la siguiente forma:
\begin{enumerate}
 \item Se aplica la regla de transición a cada una de las 8 configuraciones anteriores,
 \item Se concatenan los bits obtenidos,
 \item Se interpreta dicha concatenación como un número en base 2 y
 \item Se considera la expresión decimal de dicho número.
\end{enumerate}

El número así obtenido se denomina el \textit{número de Wolfram} de la regla de transición considerada. Un ejemplo de esta regla sería:\\

Consideremos un Autómata Celular de $n = 11$ células, en el que $k = 2$ y $r = 1$. El conjunto de estado se este autómata celular es $\mathbb{Z}$ y la vecindad ya mencionada, además la regla de transición se rige por la siguiente expresión:
$$a_1^{t+1} = \lbrace a_{i-1}^t + a_{1}^t +a_{i+1}^t \rbrace mod 2, \qquad i \geq 0.  $$
Si el estado inicial del Autómata Celular es definido por la siguiente configuración
$$C^0 = (0,0,0,0,0,1,0,0,0,0,0)$$

calculando 8 generaciones sucesivas según la función de transición considerada, se obtiene la tabla de evolución siguiente:

\begin{figure}[H]
\centering
\includegraphics[scale=0.7]{imagenes/ejemplo_1.png}
\end{figure}

Ahora si se sustituye cada elemento numérico por un cuadrado de color dependiendo de si la célula esta viva, color negro, o muerta, color blanco, tal como se hacia en el juego de la vida de \textit{Conway}, se obtiene una tabla en la que se muestra el diagrama de evolución de este Autómata Celular para 8 generaciones

\begin{figure}[H]
\centering
\includegraphics[scale=0.7]{imagenes/ejemplo_2.png}
\end{figure}

Para observar de forma más detallada la evolución de este Autómata Celular con la configuración inicial dada, se puede proceder a determinar, por ejemplo, las 50 primeras iteraciones. Llevando a cabo un proceso similar al anterior, de modo que sustituimos los 0 y 1 por cuadrados blancos y negros, obteniendo un diagrama de evolución como el de la siguiente figura:
 
\begin{figure}[H]
\centering
\includegraphics[scale=0.7]{imagenes/ejemplo_3.png}
\end{figure}


\subsection{Regla 30}
Una de las reglas, de los Autómatas Celulares binarios, más famosas y utilizadas es la regla 30, de \textit{Wolfram}, utilizada en multitud de campos como la criptografía, que explicaremos más adelante, generación de aleatorios  y también se ha utilizado como un generador de números aleatorios en Mathematica. En el  sistema de clasificación de Wolfram , la regla 30 es una regla de clase III, se presentan aperiódica y de comportamiento caótico.\\

Esta regla es de particular interés debido a que produce patrones complejos, aparentemente al azar a partir de reglas simples y bien definidas. Debido a esto, Wolfram cree que la regla 30, y los autómatas celulares, en general, son la clave para entender cómo las reglas simples producen estructuras y comportamientos complejos en la naturaleza. Por ejemplo, un patrón que se asemeja a la Regla 30 figura en el caparazón de multitud de especies de caracoles de tipo cono, Conus textile.

\begin{figure}[H]
\centering
\includegraphics[scale=0.7]{imagenes/concha.jpg}
\caption{textil conus fotografía}
\end{figure}


La Regla 30 utiliza el patrón de la siguiente figura, cómo se comportará la célula en la siguiente generación dependiendo de su vecindario actual.

\begin{figure}[H]
\centering
\includegraphics[scale=0.7]{imagenes/rule_30_1.png}
\caption{Patrón de la Regla 30}
\end{figure}

Además el siguiente patrón surge de un estado inicial en una única célula con el estado 1 (que se muestra como negro) está rodeado por las células con el estado 0 (blanco). si avanzamos 15 generaciones la Regla 30 podemos empezar a observar como la parte izquierda del diagrama de evolución del autómata celular se va repitiendo y la derecha sigue un movimiento caótico.

\begin{figure}[H]
\centering
\includegraphics[scale=0.7]{imagenes/rule_30_2.png}
\caption{Diagrama de evolución para 15 generaciones de la Regla 30}
\end{figure}

Si seguimos evolucionando el Autómata hasta la generación número cien obtendríamos el siguiente autómata:

\begin{figure}[H]
\centering
\includegraphics[scale=1]{imagenes/rule_30_3.jpg}
\caption{Diagrama de evolución para 100 generaciones de la Regla 30}
\end{figure}

Varios motivos están presentes en esta estructura, tal como la aparición frecuente de triángulos blancos y un patrón de rayas bien definido en el lado izquierdo; sin embargo, la estructura en su conjunto no tiene un patrón discernible. El número de células negras en la generación $n$ norte está dada por la secuencia $1, 3, 3, 6, 4, 9, 5, 12, 7, 12, 11, 14, 12, 19, 13,\\ 22, 15, 19, ... $ (secuencia A070952 en el OEIS ).

Como se desprende de la imagen de arriba, la regla 30 genera aparente aleatoriedad a pesar de la falta de cualquier cosa que pudiera considerarse razonablemente como entrada aleatoria. Stephen Wolfram propuso utilizar su columna central como un generador de números pseudoaleatorios (PRNG); que pasa a muchas pruebas estándar de aleatoriedad, y Wolfram utiliza esta regla en el producto Mathematica para crear números aleatorios. Aunque la Regla 30 produce aleatoriedad en muchos patrones de entrada, también hay un número infinito de patrones de entrada que resultan en patrones de repetición. El ejemplo trivial de tal patrón es el patrón de entrada única que consiste en ceros. Un ejemplo menos trivial, encontrado por Mateo de Cook , es cualquier patrón de entrada que consta de infinitas repeticiones del patrón de '00001000111000', con repeticiones opcionalmente separadas por seis más. Muchos de esos patrones más fueron encontrados por Frans Faase.

Sorber y Tomassini han demostrado que, como regla generador de números aleatorios 30 exhibe mal comportamiento cuando se aplica a todas las columnas de reglas en comparación con otros generadores basados en autómatas celulares. Los autores también han expresado su preocupación de que "Los resultados relativamente bajos obtenidos por la regla 30 pueden deberse al hecho de que hemos considerado secuencias aleatorias $N$ generado de forma simultánea, en lugar de uno solo el considerado por Wolfram."

\subsection{Regla 110}
La Regla 110, al igual que El Juego de la Vida, pertenece a la clase 4 según la clasificación de \textit{Wolfram} y exhibe un  comportamiento, que no es ni totalmente aleatorio ni completamente repetitivo.Las estructuras localizadas aparecen e interactúan de diversas maneras de aspectos complicados. En el curso del desarrollo de una nueva clase de ciencia , como asistente de investigación de Wolfram en 1994, Mateo Cook, demostró que algunas de estas estructuras eran lo suficientemente ricas como para apoyar la universalidad. Este resultado es interesante porque la regla 110 es un sistema unidimensional extremadamente simple, y difícil de diseñar para realizar un comportamiento específico. Por lo tanto, este resultado proporciona un apoyo importante a la vista de \textit{Wolfram} esa clase 4 sistemas son inherentemente propensos a ser universal. Regla 110 ha sido la base para algunas de las máquinas de Turing universales más pequeños.\\

La Regla 110 utiliza el siguiente patrón:

\begin{figure}[H]
\centering
\includegraphics[scale=0.7]{imagenes/rule_110_1.png}
\caption{Patrón para el Autómata Celular Lineal de la Regla 110}
\end{figure}

Y su diagrama de evolución para las primeras 15 generaciones son

\begin{figure}[H]
\centering
\includegraphics[scale=0.7]{imagenes/rule_110_2.png}
\caption{Diagrama de evolución con 15 generaciones de la Regla 110}
\end{figure}

Como se puede observar en el diagrama anterior este Autómata Celular solo crece hacia la izquierda y a simple vista parece que se genera repitiendo algunas formas. Este Autómata Celular es el único de todos los demás que ha podido completar la prueba de Turing completamente.

%--------------------------------------------------------------------------
%	Aplicaciones Actuales 
%--------------------------------------------------------------------------

\section{Aplicaciones Actuales} % 
Las aplicaciones de la teoría de autómatas celulares abarcan aspectos de la ciencia muy diversos como la computación e inteligencia artificial, la criptografía, el comportamiento de las moléculas de un gas, el flujo de tráfico y peatones, la evolución de la población, etc.

\subsection{Criptografía}
En la actualidad, la gran cantidad de información transmitida mediante redes de ordenadores y, en la mayoría de los casos, la necesidad de su confidencialidad, como por ejemplo: datos personales, cuentas bancarias, etc, hace necesario que esta información se transmita de manera fiable y segura. Está seguridad requiere del diseño e implementación de protocolos que garanticen la seguridad de los datos, de aquí nace la criptografía. El proceso de cifrar un mensaje consiste en transformarlo mediante un algoritmo de modo que sólo quien esté autorizado podrá invertir el proceso de cifrado para así poder obtener el mensaje original. Existen diversas formas de cifrar un mensaje y múltiples protocolos encargados de ello. Algunas de las técnicas de cifrado requieren una secuencia de bits pseudoaleatorios, para poder generar una clave de encriptado aparentemente aleatoria y sin ningún sentido pero que realmente si tiene una procedencia pre establecida y conocida. Para ello los Autómatas Celulares de Wolfram son excelentes como cifradores en flujo.\\

Consideremos los Autómatas Celulares de Wolfram definidos por las reglas de transición números 30 y 45, respectivamente, que parecen ser los que tienen mejores propiedades como generadores de bits pseudoaleatorios:
$$a_i^{t+1}  =  \lbrace a_{i-1}^t + a_i^t + a_{i+1}^t + a_i^t * a_{i+1}^t \rbrace \enspace  mod\enspace 2\enspace = a_{i-1}^t  \enspace XOR \enspace \big( a_i^t \enspace OR \enspace a_{i+1}^t \big) $$ 
$$a_i^{t+1}  =  \lbrace 1 + a_{i-1}^t + a_{i+1}^t + a_i^t * a_{i+1}^t \rbrace \enspace  mod\enspace 2\enspace = a_{i-1}^t  \enspace XOR \enspace \big( a_i^t \enspace OR \enspace \big( \enspace NOT \enspace a_{i+1}^t \big)\big) $$ 

Los diagramas de evolución de cada uno de los dos Autómatas Celulares anteriores pueden verse en las siguientes figuras:

\begin{figure}[H]
\centering
\includegraphics[scale=0.7]{imagenes/regla_30.png}
\caption{Diagrama de evolución del Autómata Celular de regla 30}
\end{figure}


\begin{figure}[H]
\centering
\includegraphics[scale=0.7]{imagenes/regla_45.png}
\caption{Diagrama de evolución del Autómata Celular de regla 45}
\end{figure}

Para este diagrama se han considerado 50 generaciones para cada autómata celular con la misma configuración inicial. Se podría pensar en utilizar estos dos Autómatas Celulares como generadores pseoudoaleatorios sin más que considerar como semilla a la configuración inicial y tomar como salida la lista de ceros y unos definida por la última configuración. Sin embargo, hacerlo así supondría que se generaría una lista de tantos bits  como bits de partida, dado que el número de células de todas las configuraciones es el mismo, por lo que este procedimiento no parece adecuado. Si lo que se desea es obtener una secuencia de bits cuya longitud sea mucho mayor que la longitud de la semilla o configuración inicial, es conveniente considerar como secuencia de salida la evolución de una de las células del Autómata. Con ello sólo haría falta conocer el estado inicial de unas pocas células e iterar el Autómata Celular tantas veces como bits se requieran.\\

Si se considera como salida del Autómata Celular la evolución de la célula central, sin considerar su estado inicial, a lo largo de 100 iteraciones, para la configuración inicial dada por:
$$(0,0,0,0,0,0,0,0,0,0,0,0,1,0,0,0,0,0,0,0,0,0,0,0,0,0)$$
se tiene la siguiente salida de 100 bits:
$$(1,0,1,1,1,0,0,1,1,0,0,0,1,0,1,1,0,0,1,0,0,1,1,1,1,0,$$
$$1,1,1,1,0,1,0,1,1,1,1,0,1,0,1,0,0,1,0,0,0,1,0,1,1,1,0,$$
$$0,0,0,0,1,0,0,1,1,0,1,1,1,1,1,0,0,1,0,0,1,0,0,1,0,0,$$
$$1,0,1,1,1,1,0,1,0,1,1,1,0,0,1,0,1,0,1,0,0)$$

Y su diagrama de evolución de la célula central, girado 90 grados, tiene la siguiente forma:

\begin{figure}[H]
\centering
\includegraphics[scale=0.7]{imagenes/regla_30_100.png}
\caption{Diagrama de evolución del Autómata Celular de regla 30 con 100 iteraciones}
\end{figure}

Se puede observar que con la configuración inicial de 25 células dada anteriormente, se ha obtenido la secuencia de 100 bits presentada en antes, si bien esta longitud podría ser mucho mayor sin más que iterara más veces la evolución del Autómata Celular. La decisión de la secuencia de bits a utilizar como salida dependerá del tipo de Autómata Celular de que se trate. Téngase en cuanta que existen Autómatas Celulares cuya evolución es muy simétrica, lo que dificulta su uso como generadores de números pseudoaleatorios.\\

El proceso para encriptar un mensaje utilizando la generación de pseudoaleatorios sería el siguiente. Primero definimos un Autómata Celular, indicamos el patrón inicial que tendrá y generamos una salida de bits deseada, como hicimos anteriormente con las 100 iteraciones del Autómata Celular. Ahora concatenamos el valor en binario de cada una de las letras del mensaje, según su código ASCII, y luego sumar, bit a bit, el mensaje obtenido con la secuencia de bits generada. Por ejemplo la palabra ``SECRETO'' utilizando el Autómata anterior sería:

\begin{figure}[H]
\centering
\includegraphics[scale=0.7]{imagenes/secreto.png}
\end{figure}

El destinatario para poder recuperar el mensaje original solo necesita concatenar el valor en binario de cada una de las letras o símbolos del criptograma recibido y sumar, bit a bit, el criptograma con la clave, realizando así una involución del mensaje.\\

La seguridad de este criptosistema está basada en la impredecibilidad de la clave, es decir, en la dificultad de poder obtener la secuencia pseudoaleatoria generada por el Autómata Celular. De ahí la importancia de que los Autómatas Celulares elegidos como generadores de bits pseudoaleatorios tengan buenas propiedades estadísticas. Hay estudios donde se comprueba y se revisa la seguridad de estos criptosistemas.\\



\subsection{Modelado de flujo de tráfico y peatones}
También conocido como el modelo de Nagel-Schreckenberg (Na-Sch) es un modelo teórico para la simulación de la autopista de tráfico utilizando Autómatas Celulares. El modelo fue desarrollado en la década de 1990 por los alemanes físicos Kai Nagel y Michael Schreckenberg. Se trata esencialmente de un Autómata Celular simple para modelar el flujo de tráfico, que puede reproducir los atascos de tráfico, es decir, muestran una desaceleración de la velocidad media del coche cuando el camino está lleno (alta densidad de coches). El modelo muestra cómo los atascos de tráfico pueden ser considerados como un fenómeno emergente o colectiva debido a las interacciones entre los coches en la carretera, cuando la densidad de los coches es alta y así que los coches están muy cerca el uno al otro en promedio.\\

En el modelo de Nagel-Schreckenberg, una carretera se divide en células . En el modelo original, estas células se alinean en una sola fila cuyos extremos están conectados de manera que todas las células forman un círculo, es decir un Autómata Celular de frontera periódica. Cada célula es o bien la carretera vacía o contiene un solo coche; es decir, no más de un coche puede ocupar una celda en cualquier momento. Cada coche se le asigna una velocidad que es un número entero entre 0 y una velocidad máxima (= 5 en trabajo original Nagel y de Schreckenberg).\\

Antes de introducir las reglas del Autómata Celular, se mencionan los parámetros relativos al modelo:
\begin{itemize}
\item Cada vehículo tiene asociada una posición $x$ en la autopista. Al ser un modelo de espacio discreto, cada célula equivale a un vehículo o a una célula vacía. Por convención, la posición es creciente, a partir de un índice 0.
\item  Cada vehículo tiene una velocidad $v$ asociada, que es un valor entero, tal que: $v \in \lbrace 0,1,...,v_{max} \rbrace$.
\item $v_{max}$ es la máxima velocidad que puede alcanzar cualquiera de los vehículos.
\item $p$ es la probabilidad de que un vehículo reduzca su velocidad aleatoriamente. Si $p=0$, el modelo se conoce como Na-Sch Determinista, de lo contrario se conoce como Na-Sch No Determinista. Esta probabilidad puede verse como la causante de congestionamientos aleatorios en un flujo de tráfico normal, causado en la realidad por alguna causa arbitraria (como una distracción del conductor antes de comenzar a acelerar).
\item  $b$ se conoce como la brecha, que es la distancia en células que separa a un vehículo de su predecesor (el vehículo inmediatamente adelante de él).
\end{itemize}

Debe aclararse que la velocidad está dada en células por unidad de tiempo, y al tratarse de un Autómata Celular discreto, el tiempo corre en unidades, por lo que hablar de una velocidad $n$ equivale a decir que un vehículo se moverá $n$ células hacia adelante en un paso de tiempo.\\

El modelo Na-Sch consta de 4 reglas para modelar el comportamiento de un vehículo cualquiera en la autopista. Estas reglas se refieren a la aceleración, frenado y movimiento de los vehículos:\\


Regla 1.- Aceleración $ v=min(v+1,v_{max})$. Es decir, si aún no se llega a la velocidad máxima, acelerar en una unidad.\\

Regla 2.- Frenado por la interacción con otros vehículos $v=min(v,b)$. Es decir, la velocidad será igual al mínimo entre la velocidad calculada en la regla 1 y la brecha con el predecesor. Esto evitará que el vehículo golpee al predecesor (el modelo Na-Sch original no modela accidentes vehiculares).\\

Regla 3.- Frenado aleatorio. Con probabilidad $p$, $ v=max(v-1,0)$. Es decir, con probabilidad $p$, si el vehículo aún no está completamente detenido, su velocidad (la calculada en la regla 2) se reduce en una unidad. Si $p=0$ (modelo Na-Sch determinista), esta regla nunca se lleva a cabo.\\

Regla 4.- Movimiento $ x=x+v$. Es decir, se actualiza la posición del vehículo con su nueva velocidad $v$ (la calculada en la regla 3).\\
 
Al tratarse de un Autómata Celular, debe recordarse que estas reglas se aplican a todos los vehículos de manera homogénea y en paralelo (al mismo tiempo).El Autómata Celular del modelo Na-Sch tendrá una rejilla unidimensional, con $L$ células. Las vecindades están dadas por la interacción de los vehículos y sus predecesores (las brechas). Los estados posibles de un vehículo son los enteros entre 0 y $v_{max}$. La función de transición del Autómata Celular está dada por las 4 reglas anteriormente descritas. \cite{trafico}
 
 
\subsection{Aplicación de Autómatas Celulares para la descripción de extremos de la precipitación}
Como se describe en el articulo publicado por la Universidad de Salamanca con el nombre ``APLICACIÓN DE AUTÓMATAS CELULARES PARA LA DESCRIPCIÓN DE EXTREMOS DE LA PRECIPITACIÓN'' una posible aplicación para los Autómatas Celulares es la descripción de la ocurrencia de extremos de precipitación. Las series temporales de precipitación presentan algunos grados de periodicidad o fluctuaciones debidas a variabilidad climática natural. Una vez filtradas estas componentes mediante técnicas espectrales, aplicamos el modelo propuesto de Autómatas Celulares a la serie residual con el fin de caracterizar las rachas secas y húmedas.
Nos Abstraeremos de la parte climatológica para centrarnos solo en la parte de configuración y utilización de los autómatas para obtener la descripción de los extremos de las precipitaciones.\\

Se utilizan dos Autómatas Celulares uno probabilístico y otro
determinístico y ambos con un espacio celular formado por 12 células, que representan los meses del año. El conjunto de estados será binario, 0 o 1. La función de transición consta de dos fases, una determinista y otra probabilística. La configuración inicial se toma a partir de valores reales previamente obtenidos. \\

Con este estudio se consiguió determinar con más de un 60 por ciento de exactitud la precisión del Autómata Celular para la descripción de extremos de la precipitación, en comparación con los datos que se tenían almacenados.

\section{Autómatas Celulares utilizando algoritmos genéticos}
Los algoritmos genéticos  son métodos de búsqueda en el espacio de soluciones de un problema basados en el mecanismo de la selección natural o la lucha por la supervivencia \cite{koza}. \\

Mediante la utilización de algoritmos genéticos las soluciones al problema se codifican como una secuencia de símbolos, como un programa genético, generando una población inicial de estas soluciones. Generación tras generación, las soluciones más aptas, que más se acercan al resultado final, sobreviven y se reproducen, haciendo cruzar entre sí sus cadenas de símbolos, introduciendo nuevos individuos en la población. Tras un número determinado de generaciones, se pretende obtener una población de soluciones con al menos un individuo capaz de llegar al resultado deseado para el problema.\\

En el mundo natural, se dice que un proceso está dirigido por una evolución cuando se satisfacen las siguientes cuatro condiciones.\\
\begin{enumerate}
\item Una entidad debe tener la capacidad de reproducirse.
\item Debe existir una población de tales entidades.
\item Debe existir algún tipo de variedad entre las entidades de dicha población.
\item Debe existir alguna diferencia en la capacidad de sobrevivir en el entorno asociada a dicha variedad.
\end{enumerate}

En la naturaleza, la variedad se manifiesta en las diferencias entre cromosomas en una población determinada. Esta variación en el genotipo se hace patente en la estructura y comportamiento del organismo, es decir, en la expresión externa del genotipo llamada ``\textit{fenotipo}'', lo que influye en la capacidad de adaptarse al medio.\\

Los individuos que mejor se adaptan tienen más probabilidades de sobrevivir y reproducirse, y, por lo tanto, de llegar a sus descendientes esas capacidades o variaciones en su genotipo que le han proporcionado un mayor grado de adaptación. A lo largo de sucesivas generaciones, la estructura y comportamiento de los individuos de la población hace que éstos se adapten perfectamente al entorno.\\

De este modo cualquier problema de adaptación puede observarse bajo el prisma genético. Este tipo de proceso evolutivo se denominó \textit{algoritmo genético}.\\

Los algoritmos genéticos son un tipo de algoritmos matemáticos altamente paralelos, que transforman un conjunto de objetos (población) representados matemáticamente cada uno de ellos mediante una cadena de símbolos (genotipo) asociada a una determinada medida de adaptación al medio, en otra población distinta, siguiente generación, mediante la aplicación del principio de supervivencia del más fuerte en términos de adaptación.\\

%%
%%%%%%%%%%%%%%%%%%%%%%%%%%%
%% SI HACE FALTA MAS TAMAÑO RELLENAR CON LA EXPLICACION
%%%%%%%%%%%%%%%%%%%%%%%%%%%%

A través de observaciones se ha podido determinar cómo la región de transición entre el orden y el caos da lugar a los comportamientos más complejos, de hecho la vida surge al borde del caos, y es la selección natural la que alcanza y sostiene este estado. De este modo de demuestra la relación directa entre complejidad y selección natural, siendo esta última la que consigue mantener dicha complejidad de los organismos en los valores adecuados. Si la complejidad se reduce el organismo no es capaz de desarrollar funciones avanzadas, si el organismo exhibe un comportamiento caótico tampoco es capaz de llevar a cabo dichas funciones y, en cualquier caso, no sobrevivirá a lo largo de las generaciones. Al igual que en los sistemas naturales, existe una relación directa entre la complejidad de los organismos artificiales y la selección artificial basada en los algoritmos genéticos, siendo éstos capaces de dirigir la evolución a conveniencia, para generar individuos aptos para la resolución de un problema, como se ha descrito con anterioridad.\\

Los Autómatas Celulares se han descrito como modelos matemáticos masivamente paralelos, extremadamente simples en su concepción, pero capaces de llevar a cabo en algunos casos conductas sumamente complejas y no predecibles a priori, a menos que se efectúe una simulación debido al enorme número de iteraciones entre las células.\\

Es, por esta circunstancia, prácticamente imposible, diseñar a mano un conjunto de reglas para que un Autómata Celular realice una tarea compleja, es decir, programar la tabla de reglas del autómata. La mayoría de los descubrimientos relativos a los comportamientos de configuraciones determinadas de Autómatas Celulares, regidos por una tabla de reglas determinada, es fruto casual de la observación de la simulación, como en el caso de los planeadores, pero en ningún caso concebida a priori por el diseñador.\\

Los Autómatas Celulares están definidos por su tabla de reglas. Esta tabla de reglas constituiría la cadena genética, substrato del trabajo de los algoritmos genéticos, que, a su vez, conservarían aquellas partes de la cadena, o nociones, útiles para llevar a cabo la tarea deseada, de generación en generación mediante la aplicación de los operadores genéticos como se ha explicado anteriormente, obteniendo nuevos individuos en los que se sintetizan las nociones de generaciones pasadas que aportan una mayor adaptación a los organismos, en este caso, a los Autómatas Celulares.\\

Puesto que la programación de la tabla de reglas de un Autómata Celular que lleve a cabo una función determinada es una tarea demasiado ardua para el ser humano, los algoritmos genéticos constituyen una herramienta ideal que puede ser empleada con éxito para obtener dicha tabla, y por tanto, el Autómata Celular que lleve a cabo una función compleja. Por lo tanto un Autómata Celular dotado de computación universal podría ejecutar en paralelo todos los programas posibles.. La idea del Autómata Celular como un sistema de computación se resume de la siguiente forma: a partir de una configuración inicial, que representa los datos de entrada, se puede efectuar cierto proceso, determinado por la tabla de reglas del Autómata Celular, para obtener una configuración final, que representa el resultado.\\

Uno de los primeros investigadores en generar Autómatas Celulares con algoritmos genéticos fue el físico \textit{Norman Packard}. Examinó la frecuencia de las reglas generadas como función del parámetro e interpretó los resultados como evidencias de dos hipótesis:
\begin{enumerate}
\item Los Autómatas Celulares capaces de llevar a cabo cálculos complejos, se sitúan cerca de valores críticos que determinan transiciones de fase entre regímenes ordenados y caóticos.
\newpage
\item La evolución tiende a seleccionar Autómatas Celulares con valores cercanos a dichos valores críticos.
\end{enumerate}

Generalmente en la experimentación de la aplicación de algoritmos genéticos a los Autómatas Celulares se ha utilizado Autómatas Celulares unidimensionales o lineales con una vecindad de 3 y dos posibles estados por célula y el cálculo a implementar ha sido el problema de clasificación de la mayoría. Este problema es un medio para estudiar cómo pueden llevarse a cabo cálculos complejos sobre grandes áreas mediante reglas que operan por iteracciones a distancias relativamente pequeñas. La configuración inicial de un conjunto de células en un autómata biestado unidimensional conforma los datos de entrada para el cálculo. El objetivo es obtener una tabla de reglas que defina un Autómata Celular capaz de clasificar correctamente la configuración inicial, haciéndola tender, tras un número finito de pasos, a todo 1 si había una mayoría de 1 en la configuración inicial, o a todo 0, si había una mayoría de 0.\\

Siendo $r_0$, el porcentaje de 1 en la configuración inicial,  si $r_0 > 0.5$ la configuración final debe ser todo 1, en caso de que $r_0 > 0.5$ la configuración final debe ser todo 0, y en caso que $r_0$ sea exactamente 0, el resultado es indeterminado. Estos Autómatas Celulares consiguieron una eficacia un poco superior al 80\%. Packard intentó aplicar el algoritmo genético al problema de clasificación de la mayoría para obtener un Autómata Celular que mejorase estos resultados con las siguiente pasos para cada generación:
\begin{enumerate}
\item Se calcula el grado de adaptación de cada Autómata Celular.
\item Se clasifica a la población en base a su grado de adaptación.
\item Una parte de los Autómatas Celulares con peor grado de adaptación se eliminan del sistema.
\item Éstos últimos son sustituidos por otros nuevos obtenidos aplicando las operaciones de cruzamiento y mutación a partir de los más optimos.
\end{enumerate}

Para evitar al convergencia prematura se introdujo un sistema que reforzase la diversidad decrementando artificialmente el grado de adaptación de aquellos individuos con tablas de reglas demasiado parecidas a otras ya existentes. Packard llegó a unos resultados que le permitieron aseveras que los Autómatas Celulares con mejor comportamiento eran aquellos con valores cercanos a las transiciones de ordenado a caótico.\\

\newpage
Sin embargo más adelante Melanie Mitchell demostraró que el desarrollo de Parckard era incorrecto o que no había documentando todo su proceso, ya que no consiguieron
\begin{wrapfigure}{l}{0.4\linewidth}
\centering
\includegraphics[scale=1]{imagenes/melani.jpg}
\caption{Melanie Mitchell}
\label{fig:melanie}
\end{wrapfigure}

 replicar sus resultados con el mismo grado de satisfacción. De esta forma se determinó que la relación entre la capacidad computacional de un Autómata Celular no es directa, aunque sí lo es la relación entre el y el comportamiento dinámico del autómata.\\
$\enspace \enspace \enspace \enspace \enspace$\\
$\enspace \enspace \enspace \enspace \enspace$\\

Melanie Mitchell \cite{melanie} utilizo otra técnica para aplicar los algoritmos genéticos a la generación de Autómatas Celulares:
\begin{enumerate}
\item  Se escoge al azar una población inicial de M cromosomas, que representan los Autómatas Celulares mediante su tabla de reglas.
\item El grado de adaptación de un Autómata Celular de la población se calcula escogiendo al azar I configuraciones iniciales sobre cuadricula e iterando el Autómata Celular sobre cada configuración inicial hasta llegar a un punto fijo o hasta consumir un máximo de ciclos de tiempo pre establecidos.
\item Se determina si la configuración final alcanzada es correcta para la tarea deseada, en este caso, si es todo 0 cuando $r_0 < 0.5$ o todo 1 cuando $r_0 > 0.5$. El grado de adaptación es la fracción de la configuración inicial correctamente clasificadas. Las configuraciones parcialmente correctas no son tomadas en cuenta.
\item Para cada generación: 
\begin{itemize}
\item Se crea un nuevo conjunto de I.
\item Se calcula el grado de adaptación de cada autómata.
\item Se ordenan según dicho grado de adaptación.
\item Se escoge un número E de Autómatas Celulares situados entre la élite, con alto grado de adaptación y se copian a la siguiente generación.
\item Los elementos M - E restantes se forman por cruzamiento entre los más aptos, y aplicando la operación de mutación con una determinada probabilidad.
\end{itemize}
\end{enumerate}

El resultado obtenido fue la generación de unos Autómatas Celulares que desarrollaron distintas estrategias para llevar a cabo la tarea, consiguiendo en el mejor de los casos un éxito 77.55\% de las clasificaciones correctas, menos que los obtenidos anteriormente por otros informáticos mucho antes.\\

Pero se siguió esta linea de avance de utilizar algoritmos genéticos para desarrollar Autómatas Celulares hasta el punto que se ha obtenido hasta un 86.3\% de éxito con la utilización de técnicas de coevolución.\\

En definitiva la metodología de programación convencional es, naturalmente, de poca utilidad para un sistema de computación basado en un Autómata Celular. El desarrollo de una nueva metodología es difícil y los algoritmos genéticos puede ser una herramienta fundamental en la búsqueda, puesto que la principal dificultad para los Autómatas Celulares es la computación en paralelo y de ello su programación. Por ello se argumentan desde varios expertos en Autómatas Celulares se debe perseguir que el autómata realice tareas concretas para mejorar su éxito y buscar Autómatas Celulares universales, como en El Juego de la Vida. Por ello la atención actual se dirige hacia el desarrollo de Autómatas Celulares que sean capaces de realizar tareas concretas pero complejas como el procesamiento de imágenes, el modelamiento del tráfico, etc.

%------------------------------------------------

\bibliography{citas} %archivo citas.bib que contiene las entradas 
\bibliographystyle{plain} % hay varias formas de citar

\end{document}



