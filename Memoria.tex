\input{preambuloSimple.tex}

%------------------------------------------------------------------------
%	TÍTULO Y DATOS DEL ALUMNO
%------------------------------------------------------------------------

\title{	
\normalfont \normalsize 
\textsc{\textbf{Modelos de Computación (2016-2017)} \\ Grado en Ingeniería Informática \\ Universidad de Granada} \\ [25pt] % Your university, school and/or department name(s)
\horrule{0.5pt} \\[0.4cm] % Thin top horizontal rule
\huge Autómatas Celulares \\ % The assignment title
\horrule{2pt} \\[0.5cm] % Thick bottom horizontal rule
}

\date{\normalsize\today} % Incluye la fecha actual

%-------------------------------------------------------------------------
% DOCUMENTO
%-------------------------------------------------------------------------

\begin{document}

\maketitle % Muestra el Título

\newpage

%-------------------------------------------------------------------------
%	Resumen introductorio (entre 5 y 15 lineas)
%----------------------------------------------------------------------
\begin{abstract}
Se ha realizado un trabajo sobre los autómatas celulares desarrollados en la época de los 60 por \textit{Von Neumann}. La teoría de los autómatas celulares ha ido cambiando durante su vida con diferentes vertientes y multitud de variaciones pero nos centraremos en solo dos versiones más a parte de la de Von Neumann.



\end{abstract}

%--------------------------------------------------------------------------
%	Introducción
%-------------------------------------------------------------------------

\section{Introducción} % 
% http://web.archive.org/web/20080907225701/http://yupana.autonoma.edu.co/publicaciones/yupana/005/autocelular/Automatas.html

%


\subsection{Que son los Autómatas Celulares}
Hola \cite{Teoria_von_neumann} hola
\subsection{Historia}


%--------------------------------------------------------------------------
%	Era de Von Neumann
%--------------------------------------------------------------------------

\section{Era de Von Neumann} % 

%--------------------------------------------------------------------------
%	Era de John Horton Conway
%--------------------------------------------------------------------------

\section{Era de John Horton Conway} % 


%--------------------------------------------------------------------------
%	Era de Stephen Wolfram
%--------------------------------------------------------------------------

\section{Era de Stephen Wolfram} % 

%--------------------------------------------------------------------------
%	Aplicaciones Actuales 
%--------------------------------------------------------------------------

\section{Aplicaciones Actuales} % 

\subsection{Modelado de flujo de tráfico y peatones}

\subsection{Modelado de evolución de células o virus}

%------------------------------------------------

\bibliography{citas} %archivo citas.bib que contiene las entradas 
\bibliographystyle{plain} % hay varias formas de citar

\end{document}



